\documentclass[12pt,twocolumn]{article}
\usepackage{ctex}

\title{分栏与分页}
\author{Morning}
\date{\today}

\begin{document}
\maketitle
有一次我有机会参加一个批量项目评审团,主办方请吃饭,同桌有两位院士,一位专业忘记了,70多岁,聊天中对很多重大体制问题颇有微词,但自己日夜辛苦做一些项目、论文和书籍的审查工作,也不会去想那些项目是否水项目,只是觉得自己应该兢兢业业辛苦工作,别人既然相信他这个院士的能力(或者头衔的力量),就把自己弄到极为辛苦,我后来想,一定不是什么有水平的第一性的科研工作,也不是领导科研团队做什么攻关的。第二位是计科领域的,年轻,过后也听说过他多次到处发言,自己成为大学校长等,但显然也没有任何有力度的原创科研。实际上整个中国计科,几十年来又有什么进入计算机科学和技术发展的国际有位的成果呢?从芯片到操作系统到后来的大数据和人工智能算法,连中国所有发展最快的”互联网“企业所用的一切工具和基础,百分百都是一路跟随拿来国外现成的。如果说30年前起步晚落后了,但计科是有很多领域可以半路创新的啊,为什么没有?问问院士们都干什么就好了。

院士在中国是荣誉的携带者,因此也就逐渐变成了利益的直接受益者。所以绝大多数院士都是一路如此”追求“个人发展进来的。那么如果以这样的”中国最高“名利位目标,仍然走在科研前沿,就也只能是少数院士仍然在亲历亲为的事情了。而他们,也许还要面对巨大的阻力,那些体制里面先天的不合理、后来发展到越来越不合理并且强大到无法推翻的东西,他们和普通科研工作者都要面对,所以,能够冲破的,就更少了。我估计,1300位,也许有不到100位还在亲自做或深入领导团队做顶级科研!然后再被体制耗费掉其中的90位这样?

所以,不需要寄希望于这个名头。中国会有不断出现的世界顶级科技成果的,那是年轻的硕士爱鼓捣并且走对了路的(是自己走对的,不是那位博导指导的)。DeepSeek算个不错的开始吧。并没有很高深,我完全明白它家做对了什么。然后,其它更加高深和更加艰巨的世界级难题就会有团队去攻,攻下来就会有诺奖了,不过也需要文化基础和坚持至少十几年去积累。但是,如果体制仍然顽固不该,题主所见现象,就仍然是主流,那么真的能够好好做真科研的力量(不仅是院士),就仍然极小,对于这么大一个国家,每年几千亿以上的科研投入,就仍然绝大部分浪费掉了,顺便还把年轻人给卷到厌学或者面对行政压力而抑郁或tiaolou呢。

有一次我有机会参加一个批量项目评审团,主办方请吃饭,同桌有两位院士,一位专业忘记了,70多岁,聊天中对很多重大体制问题颇有微词,但自己日夜辛苦做一些项目、论文和书籍的审查工作,也不会去想那些项目是否水项目,只是觉得自己应该兢兢业业辛苦工作,别人既然相信他这个院士的能力(或者头衔的力量),就把自己弄到极为辛苦,我后来想,一定不是什么有水平的第一性的科研工作,也不是领导科研团队做什么攻关的。第二位是计科领域的,年轻,过后也听说过他多次到处发言,自己成为大学校长等,但显然也没有任何有力度的原创科研。实际上整个中国计科,几十年来又有什么进入计算机科学和技术发展的国际有位的成果呢?从芯片到操作系统到后来的大数据和人工智能算法,连中国所有发展最快的”互联网“企业所用的一切工具和基础,百分百都是一路跟随拿来国外现成的。如果说30年前起步晚落后了,但计科是有很多领域可以半路创新的啊,为什么没有?问问院士们都干什么就好了。

院士在中国是荣誉的携带者,因此也就逐渐变成了利益的直接受益者。所以绝大多数院士都是一路如此”追求“个人发展进来的。那么如果以这样的”中国最高“名利位目标,仍然走在科研前沿,就也只能是少数院士仍然在亲历亲为的事情了。而他们,也许还要面对巨大的阻力,那些体制里面先天的不合理、后来发展到越来越不合理并且强大到无法推翻的东西,他们和普通科研工作者都要面对,所以,能够冲破的,就更少了。我估计,1300位,也许有不到100位还在亲自做或深入领导团队做顶级科研!然后再被体制耗费掉其中的90位这样?

所以,不需要寄希望于这个名头。中国会有不断出现的世界顶级科技成果的,那是年轻的硕士爱鼓捣并且走对了路的(是自己走对的,不是那位博导指导的)。DeepSeek算个不错的开始吧。并没有很高深,我完全明白它家做对了什么。然后,其它更加高深和更加艰巨的世界级难题就会有团队去攻,攻下来就会有诺奖了,不过也需要文化基础和坚持至少十几年去积累。但是,如果体制仍然顽固不该,题主所见现象,就仍然是主流,那么真的能够好好做真科研的力量(不仅是院士),就仍然极小,对于这么大一个国家,每年几千亿以上的科研投入,就仍然绝大部分浪费掉了,顺便还把年轻人给卷到厌学或者面对行政压力而抑郁或tiaolou呢。

有一次我有机会参加一个批量项目评审团,主办方请吃饭,同桌有两位院士,一位专业忘记了,70多岁,聊天中对很多重大体制问题颇有微词,但自己日夜辛苦做一些项目、论文和书籍的审查工作,也不会去想那些项目是否水项目,只是觉得自己应该兢兢业业辛苦工作,别人既然相信他这个院士的能力(或者头衔的力量),就把自己弄到极为辛苦,我后来想,一定不是什么有水平的第一性的科研工作,也不是领导科研团队做什么攻关的。第二位是计科领域的,年轻,过后也听说过他多次到处发言,自己成为大学校长等,但显然也没有任何有力度的原创科研。实际上整个中国计科,几十年来又有什么进入计算机科学和技术发展的国际有位的成果呢?从芯片到操作系统到后来的大数据和人工智能算法,连中国所有发展最快的”互联网“企业所用的一切工具和基础,百分百都是一路跟随拿来国外现成的。如果说30年前起步晚落后了,但计科是有很多领域可以半路创新的啊,为什么没有?问问院士们都干什么就好了。

院士在中国是荣誉的携带者,因此也就逐渐变成了利益的直接受益者。所以绝大多数院士都是一路如此”追求“个人发展进来的。那么如果以这样的”中国最高“名利位目标,仍然走在科研前沿,就也只能是少数院士仍然在亲历亲为的事情了。而他们,也许还要面对巨大的阻力,那些体制里面先天的不合理、后来发展到越来越不合理并且强大到无法推翻的东西,他们和普通科研工作者都要面对,所以,能够冲破的,就更少了。我估计,1300位,也许有不到100位还在亲自做或深入领导团队做顶级科研!然后再被体制耗费掉其中的90位这样?

所以,不需要寄希望于这个名头。中国会有不断出现的世界顶级科技成果的,那是年轻的硕士爱鼓捣并且走对了路的(是自己走对的,不是那位博导指导的)。DeepSeek算个不错的开始吧。并没有很高深,我完全明白它家做对了什么。然后,其它更加高深和更加艰巨的世界级难题就会有团队去攻,攻下来就会有诺奖了,不过也需要文化基础和坚持至少十几年去积累。但是,如果体制仍然顽固不该,题主所见现象,就仍然是主流,那么真的能够好好做真科研的力量(不仅是院士),就仍然极小,对于这么大一个国家,每年几千亿以上的科研投入,就仍然绝大部分浪费掉了,顺便还把年轻人给卷到厌学或者面对行政压力而抑郁或tiaolou呢。

有一次我有机会参加一个批量项目评审团,主办方请吃饭,同桌有两位院士,一位专业忘记了,70多岁,聊天中对很多重大体制问题颇有微词,但自己日夜辛苦做一些项目、论文和书籍的审查工作,也不会去想那些项目是否水项目,只是觉得自己应该兢兢业业辛苦工作,别人既然相信他这个院士的能力(或者头衔的力量),就把自己弄到极为辛苦,我后来想,一定不是什么有水平的第一性的科研工作,也不是领导科研团队做什么攻关的。第二位是计科领域的,年轻,过后也听说过他多次到处发言,自己成为大学校长等,但显然也没有任何有力度的原创科研。实际上整个中国计科,几十年来又有什么进入计算机科学和技术发展的国际有位的成果呢?从芯片到操作系统到后来的大数据和人工智能算法,连中国所有发展最快的”互联网“企业所用的一切工具和基础,百分百都是一路跟随拿来国外现成的。如果说30年前起步晚落后了,但计科是有很多领域可以半路创新的啊,为什么没有?问问院士们都干什么就好了。
\clearpage
院士在中国是荣誉的携带者,因此也就逐渐变成了利益的直接受益者。所以绝大多数院士都是一路如此”追求“个人发展进来的。那么如果以这样的”中国最高“名利位目标,仍然走在科研前沿,就也只能是少数院士仍然在亲历亲为的事情了。而他们,也许还要面对巨大的阻力,那些体制里面先天的不合理、后来发展到越来越不合理并且强大到无法推翻的东西,他们和普通科研工作者都要面对,所以,能够冲破的,就更少了。我估计,1300位,也许有不到100位还在亲自做或深入领导团队做顶级科研!然后再被体制耗费掉其中的90位这样?

所以,不需要寄希望于这个名头。中国会有不断出现的世界顶级科技成果的,那是年轻的硕士爱鼓捣并且走对了路的(是自己走对的,不是那位博导指导的)。DeepSeek算个不错的开始吧。并没有很高深,我完全明白它家做对了什么。然后,其它更加高深和更加艰巨的世界级难题就会有团队去攻,攻下来就会有诺奖了,不过也需要文化基础和坚持至少十几年去积累。但是,如果体制仍然顽固不该,题主所见现象,就仍然是主流,那么真的能够好好做真科研的力量(不仅是院士),就仍然极小,对于这么大一个国家,每年几千亿以上的科研投入,就仍然绝大部分浪费掉了,顺便还把年轻人给卷到厌学或者面对行政压力而抑郁或tiaolou呢。
\end{document}