% 导言区
\documentclass{article} % 文档类型
\usepackage{ctex} % 加载ctex宏包,用于中文支持
\usepackage{xeCJK} % 加载xeCJK宏包,用于XeLaTeX编译下的中文支持
\usepackage{geometry} % 设置页面布局
\geometry{a4paper, scale=0.8} % 设置纸张大小和缩放比例

% 正文区
\begin{document}

% 标题
\title{扶手的工作原理探究}
\author{作者姓名}
\date{\today}
\maketitle

% 摘要
\begin{abstract}
这里是文档的摘要部分,简要介绍文档的主要内容和目的。
\end{abstract}

% 正文内容
\section{引言}
这是文档的引言部分,用于介绍文档的背景和研究动机。

\section{主要内容}
这里是文档的主要内容部分,可以包含多个小节,详细阐述研究方法、实验结果等。

\subsection{子标题1}
在这一部分,可以进一步细分内容,详细说明相关细节。

\subsection{子标题2}
继续展开相关内容,使文档结构更加清晰。

\section{结论}
在结论部分,总结文档的主要发现和观点,提出未来的研究方向。

% 参考文献
\begin{thebibliography}{99}
\bibitem{ref1} 参考文献1的详细信息。
\bibitem{ref2} 参考文献2的详细信息。
\end{thebibliography}

\end{document}